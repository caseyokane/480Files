%   Casey O'kane - casey.okane@uky.edu
%   Joe Weisbrod - jtwe226@uky.edu
%   Tristan Cheam - tlcheam01@uky.edu
%   EE480 - Assignment 3: A Faster IDIOT
%   note.tex : Implementor's Note
%   Version:
%       02-14-2016 : initial


\documentclass[conference]{IEEEtran}
\usepackage{graphicx}
\usepackage{float}
\usepackage{verbatim}
\usepackage{adjustbox}

\begin{document}
\title{Assignment 3: A Faster IDIOT\\Implementor's Notes}
\author{\IEEEauthorblockN{Casey O'Kane}
        \IEEEauthorblockA{casey.okane@uky.edu}
        \IEEEauthorblockN{Joe Weisbrod}
        \IEEEauthorblockA{jtwe226@uky.edu}
        \IEEEauthorblockN{Tristan Cheam}
        \IEEEauthorblockA{tlcheam01@uky.edu}}

\maketitle

\begin{abstract}
The goal of this assignment involved implementing the pipelined version of 
the IDIOT instruction set using the AIK assembler, the Verilog Hardware
Design Language and detailed test plan to exhaustively test the different 
components and logic of the design. 
\end{abstract}

\section{General Approach}
Following the approach of previous assignments, the implementation of this 
assignment closely followed a heavily modulated design. Both the data and 
instruction memories where treated as modules along with the ALU incrementor,
ALU module (for ALU instructions), the dependency detection module, the 
register file and the processor itself. 

Similar to the previous assignment, global constants were declared both to 
represent common data structures (like 'WORD) and the different Opcode 
signals that could be received by the processor (like 'OpAdd or any of the 
instructions specified by the IDIOT ISA). Also, the AIK specification IDIOT\_SPEC
provided by Dr. Dietz following the submission of Assignment 2 was used
for the sake of consistency.

The main point of difference for this assignment is the addition of the three
buffer modules that are used to share information between modules of the 
different cycles (Instruction Fetch, Register Read, ALU/Memory, Register
Write). These buffers are used to implement the pipelined design and their 
implementations are described in the next section.

While there are still some incomplete aspects of this assignment (these issues 
are discussed in the \textbf{Issues} section of these notes), the most recent 
diagram is shown in \textbf{Appendix A} at the end of these notes. 

%%%%%%%%%%%%%%%%%%%%
\section{Implementation}
This section describes how each module was implemented.

\subsection{Pipeline Buffers}
\subsubsection{Instruction Buffer}

\subsubsection{Register Buffer}

\subsubsection{ALU/Write Buffer}

\subsection{Proccesor}

\subsection{Memory}
\subsubsection{Instruction Memory}

\subsubsection{Data Memory}

%%%%%%%%%%%%%%%%%%%%%%

\section{Testing}

\subsection{Verilog Modules}

\subsection{Testing Results}

%%%%%%%%%%%%%%%%%%%%%%%
\section{Issues}
\subsection{Features Not implemented}
\subsubsection{Jump Based Instructions}
Currently, there is no implementation concerning Jump based instructions JZ/SZ/SYS. 
Hopefully this is something that will change prior to submission. 
\subsubsection{Memory based instruction}
There is no implementation of Load or store 
\subsubsection{Load Immediate Instruction} 
As the group was unable to account for the jump or memory based instructions until far later 
than anticipated, the group was unable to complete li instruction handling for the pipelined
processor. 

\subsection{Known Errors}
As it currently stands there are no known issues

\end{document}
