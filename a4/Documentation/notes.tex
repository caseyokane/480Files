%   EE480 - Assignment 4: Floating
%   note.tex : Implementor's Notes


\documentclass[conference]{IEEEtran}
\usepackage{graphicx}
\usepackage{float}
\usepackage{verbatim}
\usepackage{adjustbox}

\begin{document}
\title{Assignment 4: Floating\\Implementor's Notes}
\author{\IEEEauthorblockN{Abu-Taha Abdulfattah}
        \IEEEauthorblockA{tat\_489@hotmail.com}
        \IEEEauthorblockN{Casey O'kane }
        \IEEEauthorblockA{casey.okane@uky.edu}
        \IEEEauthorblockN{LiangLiang Zheng}
        \IEEEauthorblockA{lzh229@g.uky.edu}}

\maketitle

\begin{abstract}
The goal of this assignment involved implementing the floating point based
instructions of the IDIOT instruction set. The newly adapted ALU was then
interfaced with the provided pipeline solution and a series of tests were created
to ensure module and interface correctness.
\end{abstract}

\section{General Approach}
Include general approach here

%%%%%%%%%%%%%%%%%%%%
\section{Implementation}
This section describes how each instruction was implemented.

\subsection{Pipeline Solution}

\subsection{Floating Point Instructions}
\subsubsection{Integer to Float}

\subsubsection{Float to Integer}

\subsubsection{Float Addition}

\subsubsection{Float Multiply}

\subsubsection{Float Inverse}

%%%%%%%%%%%%%%%%%%%%%%

\section{Testing}

\subsection{Verilog Modules}

\subsection{Testing Results}

%%%%%%%%%%%%%%%%%%%%%%%
\section{Issues}
\subsection{Features Not implemented}


\subsection{Known Errors}
As it currently stands there are no known issues

\end{document}
